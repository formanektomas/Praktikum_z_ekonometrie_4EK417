%%%%%%%%%%%%%%%%%%%%%%%%%%%%%%%%%%%%%%%%%
% Beamer Presentation
% LaTeX Template
% Version 1.0 (10/11/12)
%
% This template has been downloaded from:
% http://www.LaTeXTemplates.com
%
% License:
% CC BY-NC-SA 3.0 (http://creativecommons.org/licenses/by-nc-sa/3.0/)
%
%%%%%%%%%%%%%%%%%%%%%%%%%%%%%%%%%%%%%%%%%

%----------------------------------------------------------------------------------------
%	PACKAGES AND THEMES
%----------------------------------------------------------------------------------------

\documentclass{beamer}

\mode<presentation> {
\usetheme{Madrid}
\usefonttheme{serif} 
\setbeamertemplate{navigation symbols}{} % To remove the navigation symbols from the 
}
\usepackage{lmodern}  
\usepackage{graphicx} % Allows including images
\usepackage{booktabs} % Allows the use of \toprule, \midrule and \bottomrule in tables
\usepackage[T1]{fontenc}
\usepackage[utf8]{inputenc}
\usepackage{amsmath}
\usepackage{caption} 
\usepackage{color}
\usepackage{xcolor}
\usepackage[czech]{babel}
\usepackage{lmodern}  
\usepackage{rotating}
\usepackage{scrextend}
\usepackage{pifont}
\usepackage{hyperref}
\usepackage{bm}
\usepackage{tikz}
\usetikzlibrary{arrows,positioning}
\usetikzlibrary{calc}
%
\newcommand*\circled[1]{\tikz[baseline=(char.base)]{
    \node[shape=circle,draw=red,inner sep=2pt] (char) {#1};}}
%
\newcommand*\circledd[1]{\tikz[baseline=(char.base)]{
    \node[shape=circle,draw=ProcessBlue, dashed, inner sep=2pt] (char) {#1};}}
%
\newcommand{\mytikzmark}[2]{%
  \tikz[remember picture,inner sep=0pt,outer sep=0pt,baseline,anchor=base] 
    \node (#1) {\ensuremath{#2}};}
%
%
\newcommand*{\boxcolor}{Red}
\makeatletter
\renewcommand{\boxed}[1]{\textcolor{\boxcolor}{%
\tikz[baseline={([yshift=-1ex]current bounding box.center)}] \node [rectangle,semithick, minimum width=1ex,draw, dashed] {\normalcolor\m@th$\displaystyle#1$};}}
 \makeatother
%
%------------------------------
%	TITLE PAGE
\title[Block 6]{Praktikum z ekonometrie} 
\author{VŠE Praha} 
\institute[4EK417] 
{
% Your institution for the title page
\medskip
\textit{Tomáš Formánek} % Your email address
}
\date{} % Date, can be changed to a custom date
%------------------------------
\begin{document}
\begin{frame}
\titlepage % Print the title page as the first slide
\end{frame}
%------------------------------
\begin{frame}
\frametitle{Block 6 – Treatment effects – Outline
} 
\tableofcontents 
\end{frame}
%------------------------------
\section{Treatment effects: Introduction}
\begin{frame}{Treatment effects: Introduction}
\begin{itemize}
    \item \textbf{Treatment effect analysis:} to evaluate the impact of intervention (treatment) on some \textbf{outcome} of interest.
    \bigskip
    \item Response to treatment is evaluated relative to a benchmark:  \\no treatment (control) or different treatment. 
    \bigskip
    \item Analysis of treatment effect is typically based on regression models, with outcome as the dependent variable.
    \bigskip
    \item Treatment effect analysis is generally based on the framework of Rubin's causal model. 
\end{itemize}
\end{frame}
%------------------------------
\begin{frame}{Treatment effects: Introduction}
\textbf{Examples}\\ \medskip
\begin{itemize}
    \item Wage effect of enrollment in a skill-training program.
    \medskip
    \item Health effects (speed of recovery), if new drug / medical procedure is used.
    \medskip
    \item Student performance upon being educated in small classes as opposed to large classes (being in a small class is the treatment). 
\end{itemize}
\bigskip
\textbf{Key topics of the analysis:}\\
Treatment participation: random assignment or self selection?\\
Treatment effects: actual effects or influence by confounding factors?\\
\bigskip
\textbf{Multiple treatments:} If treatment varies in intensity or type. Same principle of analysis, the choice of benchmark is more flexible.
\end{frame}
%------------------------------
% Define box and box title style
\tikzstyle{mybox} = [draw=blue!35, fill=white, very thick,
    rectangle, rounded corners, inner sep=1pt, inner ysep=10pt]
\tikzstyle{fancytitle} =[fill=blue!35, text=black]
%---------------------------------
\begin{frame}{Treatment effects: Basic notation \& terminology}
\begin{itemize}
    \item Every $i$th individual in a population has a potential outcome $y_i$ and can be exposed to treatment $C_i=\{0;1\}$. 
    \medskip
    \item $y_{i1} = y_i | (C_i = 1) $ for the treated, and \\
    $y_{i0} = y_i | (C_i = 0)$ for the non-treated.
    \medskip
    \item Average treatment effect (averaged across population):
    $$\textnormal{ATE} = E \left[ y_{i1}-y_{i0}\right],$$
    and the $i$th observation only exist in one of the two states.
    \medskip
    \item \textbf{Average treatment effect of the treated} is more of interest:
    $$\textnormal{ATET} = E \left[ y_{i1}-y_{i0} | C_i = 1\right],$$
    and the second term $y_{i0}$ is a missing counterfactual. 
    \medskip
    \item Individuals will only exist in one state: treated/untreated. Multiple assumptions apply for ATE/ATET estimation.
\end{itemize}  
\end{frame}
%------------------------------
\begin{frame}{Treatment effects: Study \& data types}
\textbf{Types of studies and data used for TE analysis}:\\
\begin{itemize}
    \item Scientific experiments under randomisation: assignment into treated and control groups is random. Relatively rare in socio-economic studies. 
    \medskip
    \item Observational studies (natural experiments, quasi-experiments): assignment into treatment and control group is not random. \\ \medskip
\end{itemize}
\bigskip
\textbf{Main problems of TE analysis:}
\begin{itemize}
    \item \textbf{Endogeneity of treatment} and self-selection bias: if treatment participation is optional, individuals who choose to participate may be systematically different from non-participants.
    \medskip
    \item \textbf{Missing counterfactual:}  Individuals always measured as either treated or untreated, we cannot observe both $y_{i1}$ and $y_{i0}$.
\end{itemize}
    

\end{frame}
%---------------------------------
\begin{frame}{Treatment effects: Study \& data types}
\begin{tikzpicture}
\node [mybox] (box){%
\begin{minipage}{0.50\textwidth}
\includegraphics[width=\textwidth, height=2.89cm]{./IMG/Obrazek1}
\begin{itemize}
\scriptsize
\item Test tubes identical except for catalyst
\item Measure: Effect at different catalyst volumes (reaction speed, product volume, \dots)
\item Perform the experiment $n$-times
\item Control for other factors (heat, \dots)
\item Estimate average effects \& standard errors
\end{itemize}
\end{minipage}
};
\node[fancytitle, right=5pt,  rounded corners] at (box.north west) {\scriptsize Scientific experiment};
\end{tikzpicture}%
\begin{tikzpicture}
\node [mybox] (box){%
\begin{minipage}{0.50\textwidth}
\includegraphics[width=\textwidth]{./IMG/Obrazek2}
\begin{itemize}
\scriptsize
\item Garbage incinerator is built in one given
suburban area over time
\item How do we estimate the effect on
individual house-prices?
\item Identical control group does not exist…
\item Different estimators exist -- multiple
assumptions apply!
\end{itemize}
\end{minipage}
};
\node[fancytitle, right=5pt,  rounded corners] at (box.north west) {\scriptsize Natural experiment (quasi-experiment)};
\end{tikzpicture}%
\end{frame}
%---------------------------------
\begin{frame}{Treatment effects: Three main approaches to analysis}
Types of analysis, key assumptions \& requirements:\\
\bigskip
\begin{itemize}
    \item \textbf{Differences in differences (DiD) estimator}\\
        \begin{itemize}
            \item Independence of treatment on the outcome at the base ($y_{i0}$), \\i.e. treatment assignment exogenous/random
            \item Parallel trends assumed
        \end{itemize}
    \bigskip
    \item \textbf{Propensity score matching (PSM):}
        \begin{itemize}
            \item Unconfoundedness (treatment assignment as good as random \\after accounting for covariates)
            \item Common support (for the treated and untreated)
            \item Large sample 
            \item PS - essentially the probability of treatment participation
        \end{itemize}
    \bigskip
    \item \textbf{Regression discontinuity design (RDD):}
        \begin{itemize}
            \item Assignment variable \& assignment threshold exist
            \item Local randomization assumed
        \end{itemize}
\end{itemize}
\end{frame}
%---------------------------------
\begin{frame}{Treatment effects: Unconfoundedness}
\begin{center}
\begin{tikzpicture}[
    box/.style={rectangle, draw, minimum size=0.6cm},
    arrow/.style={thick,->,>=stealth}
]
\node[box] (box4) at (-3,0) {Exposure};
\node[box] (box5) at (3,0) {Outcome};
\draw [dotted, ->] (box4) -- (box5) node[midway, above, red] {Treatment effect};
\end{tikzpicture}
\end{center}
%
\begin{center}
\begin{tikzpicture}[
    box/.style={rectangle, draw, minimum size=0.6cm},
    arrow/.style={thick,->,>=stealth}
]
\node[box] (box1) at (0,0) {Confounder};
\node[box] (box2) at (-3,-0.6) {Exposure};
\node[box] (box3) at (3,-0.6) {Outcome};
\draw[arrow] (box1) -> (box2);
\draw[arrow] (box1) -> (box3);
\draw [dotted, ->] (box2) -- (box3) node[midway, red] {$\times$};
\end{tikzpicture}
\end{center}
Often: combination of actual TE \& confounding factor influence.\\
\bigskip
\small  
\textbf{Example:} Company has two types of trucks, A and B (using the new A-Truck is the treatment). We want to compare fuel efficiency (outcome). Using MPG, we find A-Trucks are more efficient. However, A-Trucks often drive on highways, while B-Trucks are mostly in the city. This route difference is a confounding variable, making our results unreliable as highway driving is generally more fuel-efficient. \\
\smallskip
To improve the study, we could randomize truck assignments for equal city and highway driving, introduce city driving as another independent variable to a model, or segment the study into city and highway driving comparisons.
\end{frame}
%---------------------------------
\begin{frame}{Treatment effects: Unconfoundedness}
\textbf{Unconfoundedness} states that the potential outcomes are independent of the treatment assignment, conditional on a set of observed covariates. Once we control for these covariates, the treatment \emph{assignment} does not provide any additional information about the \emph{potential} outcomes:
$$
(y_0, y_1) \perp  \textit{Tr} ~|~ \bm{X},
$$
- $(y_0)$ and $(y_1)$ are the potential outcomes under control and treatment,\\
- $\textit{Tr}$ is the treatment assignment (dummy variable), \\
- $\bm{X}$ is the set of observed covariates. \\
\bigskip
\textbf{Independence of treatment on the outcome at the base (for the untreated)} is a stronger assumption. Here, the potential outcome under control is independent of the treatment assignment, even without conditioning on covariates:
$$
y_0 \perp \textit{Tr}
$$
\end{frame}
%---------------------------------
\begin{frame}{Treatment effects: Unconfoundedness}
\textbf{Unconfoundedness} assumption allows for the possibility that treatment assignment might be related to the potential outcomes through the observed covariates. Once we control for these covariates, the treatment assignment is assumed to be as good as random. \\
\bigskip
\textbf{Independence of treatment on the outcome} assumption requires the treatment assignment to be as good as random without conditioning on any covariates. This is a stronger assumption and is less likely to hold in real-world situations.
\end{frame}
%---------------------------------
\begin{frame}{Treatment effects: Unconfoundedness \& \texttt{do()} notation}
\small 
$$P(Y|\texttt{do}(X)) = P(Y)$$ 
LHS can be red as: ``the probability of $Y,$ given that you do  $X$''. The expression above describes the case where $Y$ is independent of doing $X$.\\
\bigskip
Say, we have an outcome $y$, a dummy treatment variable $\textit{Tr}$ and potential counfounding variable $z$ that influences both $y$ and $\textit{Tr}$. If unconfoundedness holds, the following equation holds
$$P(y | \texttt{do}(\textit{Tr})) = P(y|\textit{Tr})$$
for all values $\textit{Tr}$ and $y$, where $P(y|\textit{Tr})$ is the conditional probability. This equality states that $\textit{Tr}$ and $y$ are not confounded whenever the observationally witnessed association between them is the same as the association that would be measured in a controlled experiment, with $\textit{Tr}$ randomized. Otherwise, we have to account for the confounding factor $z$:
$$P(y | \texttt{do}(\textit{Tr})) = \sum_z P(y|\textit{Tr},z)P(z). $$
\centering
\url{https://en.wikipedia.org/wiki/Confounding}
\end{frame}
%---------------------------------
\begin{frame}{Treatment effects analysis}
\textbf{Estimation approaches for TE analysis:}\\
\bigskip
\begin{enumerate}
    \item Differences in differences (DiD)
    \bigskip
    \item \textcolor{lightgray}{Propensity score matching (PSM)
    \bigskip
    \item Regression discontinuity design (RDD)}
\end{enumerate}
\end{frame}
%---------------------------------
\section{Differences in differences (DiD) estimator}
\begin{frame}{DiD: Introduction \& example}
\vfill
{\footnotesize \textbf{Treatment: employee training for women returning from maternal leave\\ Outcome: wage effect}} \\
\medskip
\includegraphics[width=\textwidth]{./IMG/Obrazek3}
\end{frame}
%---------------------------------
\begin{frame}{DiD estimator: Assumptions}
\begin{itemize}
    \item \emph{Exchangeability (parallel trends)}: in the absence of treatment, the average outcomes for the treated and control groups would follow the same trend over time.
    \item \emph{Positivity}: every individual (unit) has a positive probability of receiving the treatment.
    \item \emph{Stable unit treatment value assumption (SUTVA)}: the potential outcomes for any individual do not depend on the treatment assignment of the other individuals.
    \item \emph{No spillover effects}: the treatment of one individual does not affect the outcome of another individual.
    \item \emph{Treatment unrelated to outcome at baseline}: the allocation of treatment is not determined by the outcome. 
    \item \emph{Stable composition of treatment and control groups}: applies to repeated treatments/interventions.
\end{itemize}
\end{frame}
%---------------------------------
\begin{frame}{DiD: Model motivation}
\small
With cross-sectional data and \textbf{exogenous}  treatment variable $D = \{0;1\}$, \\we can formulate (estimate, interpret) a regression model such as:
$$
y_i = \bm{x}_i^{\prime}\bm{\beta} + \delta D_i + \varepsilon_i.
$$\\
\medskip
\textbf{DiD estimator} refers to panel-based (or using pooled CS data) analysis, based on two dummy variables and their interaction: $T$ differentiates between two time periods (before/after treatment) and $D$ distinguishes the two groups (treatment/control):
$$y_{it}=\beta_0 + \beta_1 D_i + \beta_2 T_t + \delta_1 (D_i T_t) + \bm{x}_i^{\prime}\bm{\beta} + \varepsilon_{it}.$$

\end{frame}
%---------------------------------
\begin{frame}{DiD estimator: Model description} 
\medskip 
$$y_{it}=\beta_0 + \beta_1 D_i + \beta_2 T_t + \delta_1 (D_i T_t) + \varepsilon_{it},$$
where:
\begin{itemize}
\item $i=1, \dots, N;~~t=1,2$. \\
\item[$T_t$] is a time dummy, $T_1=0$ is the first (pre-treatment) period and \\$T_2 = 1$ is the second period (post treatment),
\item[$D_i$] is a treatment dummy, $D_i=1$ for the treated,
\item[$D_i T_t$] is an interaction element, i.e. $(D_i\! \cdot \! T_t)$,
\item[$\delta_1$] is the DiD estimator (coefficient). 
\end{itemize}
\bigskip
For simplicity, the term $\bm{x}_{it}^{\prime}\bm{\beta}$ is removed here. Note that adding $\bm{x}_{it}^{\prime}\bm{\beta}$ back to the model doesn't change the general interpretation of $\delta_1$.
\end{frame}
%---------------------------------
\begin{frame}{DiD estimator: Interpretation of the DiD $\delta_1$ coefficient}
$$y_{it}=\beta_0 + \beta_1 D_i + \beta_2 T_t + \delta_1 (D_i T_t) + \varepsilon_{it}$$\\
\bigskip
\footnotesize
\begin{table}
\captionsetup{labelformat=empty}
\centering
\caption{Table: Illustration of the DiD estimator}\label{Tab1}
\begin{tabular}{|l|c|c|c|}
\hline
\multicolumn{1}{|c|}{$E(y_{it} | D_i, T_t)$} & Before $(t = 1)$    & After $(t=2)$                             & After -- Before        \\ \hline
Control $(D_i=0)$                                    & $\beta_0$           & $\beta_0 + \delta_0$                      & $\delta_0$            \\ \hline
Treatment $(D_i=1)$                                  & $\beta_0 + \beta_1$ & $\beta_0 + \delta_0 + \beta_1 + \delta_1$ & $\delta_0 + \delta_1$ \\ \hline
Treatment -- Control                        & $\beta_1$           & $\beta_1 + \delta_1$                      & \circled{$\delta_1$}            \\ \hline
\end{tabular}
\end{table}

~\\
Again, if $\bm{x}_{it} \bm{\beta}$ is added back to the equation, interpretation of $\delta_1$ remains \\essentially unchanged.

\end{frame}
%---------------------------------
\begin{frame}{DiD estimator: Interpretation of the DiD $\delta_1$ coefficient}
$$y_{it}=\beta_0 + \beta_1 D_i + \beta_2 T_t + \delta_1 (D_i T_t) + \varepsilon_{it},$$\\
\bigskip
By rearranging an estimated model, we can express $\delta_1$ as follows:
\begin{align*}
\hat{\delta}_1 &= (\overline{y}_{Tr,\,t=2} - \overline{y}_{Co,\,t=2}) -  (\overline{y}_{Tr,\,t=1} - \overline{y}_{Co,\,t=1}), \\ ~& \\
& \hspace{0.5cm} \textnormal{which may be also rearranged as:} \\ ~& \\
&= (\overline{y}_{Tr,\,t=2} - \overline{y}_{Tr,\,t=1}) -  (\overline{y}_{Co,\,t=2} - \overline{y}_{\textit{Co},\,t=1}),
\end{align*}
where \\ \textit{Tr} subscript stands for treatment group, and \\ \textit{Co} subscript identifies the control group.
\end{frame}
%---------------------------------
\begin{frame}{DiD estimator: Example}
\footnotesize{What is the effect of building garbage incinerator on housing prices?}
\scriptsize
\begin{table}[]
\centering
\label{Tab21}
\begin{tabular}{lclcc}
\multicolumn{3}{l}{Dependent Variable: RPRICE}                                          &                      & \multicolumn{1}{l}{}      \\
\multicolumn{3}{l}{Included observations: 321}                                          &                      & \multicolumn{1}{l}{}      \\
                                &                      & \multicolumn{1}{c}{}           &                      & \multicolumn{1}{l}{}      \\
\multicolumn{1}{c}{Variable}    & Coefficient          & \multicolumn{1}{c}{Std. Error} & t-Statistic          & \multicolumn{1}{l}{Prob.} \\
                                
                                & \multicolumn{1}{l}{} &                                & \multicolumn{1}{l}{} & \multicolumn{1}{l}{}      \\
\multicolumn{1}{c}{(Intercept)}           & 82517.23             & \multicolumn{1}{c}{2726.910}   & 30.26034             & 0.0000                    \\
\multicolumn{1}{c}{Y81}         & 18790.29             & \multicolumn{1}{c}{4050.065}   & 4.639502             & 0.0000                    \\
\multicolumn{1}{c}{NEARINC}     & -18824.37            & \multicolumn{1}{c}{4875.322}   & -3.861154            & 0.0001                    \\
\multicolumn{1}{c}{Y81*NEARINC} & -11863.90            & \multicolumn{1}{c}{7456.646}   & -1.591051            & 0.1126                    \\
                                
                                & \multicolumn{1}{l}{} &                                & \multicolumn{1}{l}{} & \multicolumn{1}{l}{}      \\
R-squared                       & 0.173948             & \multicolumn{2}{l}{Mean dependent var}                & 83721.36                  \\
Adjusted R-squared              & 0.166131             & \multicolumn{2}{l}{S.D. dependent var}                & 33118.79                  \\
S.E. of regression              & 30242.90             & \multicolumn{2}{l}{Akaike info criterion}             & 23.48429                  \\
Sum squared resid               & 2.90E+11             & \multicolumn{2}{l}{Schwarz criterion}                 & 23.53129                  \\
Log likelihood                  & -3765.229            & \multicolumn{2}{l}{Hannan-Quinn criter.}              & 23.50306                  \\
F-statistic                     & 22.25107             & \multicolumn{2}{l}{Durbin-Watson stat}                & 1.557107                  \\
Prob(F-statistic)               & 0.000000             & \multicolumn{2}{l}{}                                  & \multicolumn{1}{l}{}     
\end{tabular}
\end{table}
\begin{tikzpicture}[<-,overlay,remember picture,inner sep=1.5pt,shorten <=0.2em,font=\footnotesize]
\tikzset{
    mynode/.style={rectangle,draw=blue, fill=blue!30, very thick, inner sep=.5em, minimum size=2em, text width=25em}
}
\node[mynode] at (7.8, 0.3) (Table){
\scriptsize{RPRICE - house price in real terms (USD) \\ 
$Y81$ – dummy variable for $1981$, \ ($t=1978,1981$),\\
~~$1978$ – before ``rumors''; $1981$ – incinerator operational \\
NEARINC – dummy for the treatment group}};
\end{tikzpicture}
\end{frame}
%---------------------------------
\begin{frame}{DiD estimator: selection bias}
\begin{block}{Selection bias (treatment effect vs. selection bias):}
\small
\textbf{Incinerator example:} Say, we have a “poor neighborhood” with relatively old and small houses and low house-prices. For complex reasons, it suffers from a representation deficit within the local city council (as compared to other “rich neighborhoods”) and is therefore more likely to get the incinerator. To address this problem, we would need variables to control for this factor.\\
\medskip
\textbf{Job training example:} For a natural experiment with job-training effects, we have voluntary participation. If more agile workers tend to participate more, we cannot assume treatment and control group are identical -- except for the treatment. Hence, besides of treatment effect, DiD would reflect latent ``propensity'' to participate.
\end{block}
\medskip 
DiD will be biased and inconsistent in both cases.
\end{frame}
%---------------------------------
\section{Propensity score matching}
\begin{frame}{Treatment effects analysis}
\textbf{Estimation approaches for TE analysis:}\\
\bigskip
\begin{enumerate}
    \item \textcolor{lightgray}{Differences in differences (DiD)}
    \bigskip
    \item Propensity score matching (PSM)
    \bigskip
    \item \textcolor{lightgray}{Regression discontinuity design (RDD)}
\end{enumerate}
\end{frame}
%---------------------------------
\begin{frame}{Propensity score matching: Matching vs. PSM}
\centering
\includegraphics[trim = 0cm 0cm 0cm 6cm, clip, width=0.5\textwidth]{./IMG/PSM1.jpg}
\\
\medskip
\includegraphics[trim = 0.1cm 0.15cm 0cm 0.15cm, clip, width=0.3\textwidth]{./IMG/PSM2.jpg}
\end{frame}
%---------------------------------
\begin{frame}{PSM: Assumptions and features}
\begin{itemize}
    \item \emph{Unconfoundedness:} assignment to the treatment is independent of the potential outcomes given a set of observed covariates.
    \item \emph{Common Support} or overlap condition, meaning that individuals with the same characteristics have a positive probability of being both in the treatment and control group.
    \item \emph{Large sample sizes:} PSM often requires large sample sizes because matching may not be possible if the overlap between treatment and control groups is not sufficient.
    \bigskip
    \item \emph{Matching}: PSM involves pairing treated and untreated subjects who have similar propensity scores (PS -- estimated probability of treatment participation).
    \item \emph{Propensity score} is a balancing score. Conditional on the propensity score, distribution of observed baseline covariates will be similar between treated and untreated subjects.
\end{itemize}
\end{frame}
%---------------------------------
\begin{frame}{PSM: Main steps}
\begin{enumerate}
    \item Collect data, identify that PSM is viable and appropriate
        \begin{itemize}
            \item Check assumptions
            \item Basic analysis using non-matched data
        \end{itemize}
    \item Calculate propensity scores
        \begin{itemize}
            \item Logit/probit with exposure (treatment) as dependent variable
            \item Include all predictors of the exposure and none of effects of the exposure. Confounders and interaction variables can be included.
        \end{itemize}
    \item Match subjects on the propensity scores
        \begin{itemize}
            \item 1-to-1 matching, 1-to-$n$ matching
            \item Nearest neighbor, Caliper (type of NN), Kernel, Radius, etc.
        \end{itemize}
    \item Assess quality of the matching\\
        \begin{itemize}
            \item Substantial overlap in covariates between treated \& control groups
            \item Use diagnostic metrics (e.g. standardized difference)
        \end{itemize}
    \item Analyze the propensity-matched data
        \begin{itemize}
            \item Multiple regression models (use treatment as model variable) 
            \item DiD on matched data, survival analysis, etc.
        \end{itemize}
\end{enumerate}
\end{frame}
%---------------------------------
\begin{frame}{PSM-based analysis}
\begin{enumerate}
    \item[1] Basic analysis on non-matched data -- Difference-in-means for:\\ \medskip
    \begin{itemize}
        \item Outcome variable (before \& after treatment)
        \item Regressors (covariates)
        \item Regressors (before \& after treatment)
    \end{itemize}
    \bigskip
    \item[2] Propensity score calculation\\ \medskip
    \begin{itemize}
        \item Estimate the propensity score by logit (probit) model
        \item Predicted values $\hat{y}_i$ (on the response scale) are the propensities to treatment (expected probabilities of treatment participation)
        \item Evaluate common support (e.g. by plotting histograms of estimated propensities by treatment status)     
    \end{itemize}
\end{enumerate}
\end{frame}
%---------------------------------
\begin{frame}{PSM-based analysis}
    \centering
    PS -- Common support evaluation example \\(Going to catholic school is the treatment)\\
\includegraphics[trim = 0cm 0cm 0cm 0cm, clip, width=0.7\textwidth]{./IMG/PSM_comm_supp.png}    
\end{frame}
%---------------------------------
\begin{frame}{PSM-based analysis}
    \centering
    PS -- Common support evaluation example \\(General case)\\
\includegraphics[trim = 0cm 0cm 0cm 0cm, clip, width=0.7\textwidth]{./IMG/PSM_comm_supp_2.jpg}    
\end{frame}
%---------------------------------
\begin{frame}{PSM-based analysis}

\textbf{Simple method for estimating the treatment effect:}\\
(based on propensity scores)\\ \medskip
\begin{itemize}
    \item Restrict the sample to observations within the region of common support (based on estimated PS values).
    \item Divide the sample within the region of common support into 5 quintiles, based on the estimated propensity scores.
    \item For each of these 5 quintiles, estimate the mean difference in outcome variable by treatment status. 
    \item Rubin and others suggest that this is sufficient to eliminate 95\% of the bias due to confounding of treatment status with a covariate.
\end{itemize}
\end{frame}
%---------------------------------
\begin{frame}{PSM-based analysis}
\begin{enumerate}
    \item[3] \textbf{Propensity score-based matching} \\ \smallskip We seek to find pairs of observations that have very similar propensity scores, but that differ in their treatment status. \\
    \smallskip
    \textit{Many PSM-based methods exist}, we focus on common approaches that fit most types of data.
\end{enumerate}
\begin{itemize}
    \item \textbf{Nearest neighbor:} for each $i$th treated person, the untreated with the closest propensity score is selected as the match using\\ the expression
    $$
    \underset{j}{\min}~|\hat{y}_i - \hat{y}_j|,
    $$
    where $\hat{y}_i$ is the PS for the $i$th participant (treated) and $\hat{y}_j$ is the PS of the $j$th non-participant (untreated). 
\end{itemize}
\end{frame}
%---------------------------------
\begin{frame}{PSM-based analysis}
\begin{itemize}
    \item \textbf{Caliper} is a variation of nearest neighbor matching: the $j$th non-participant is only selected as a match to the $i$th participant, if the PS distance is within the caliper limit 
    $$
    |\hat{y}_i - \hat{y}_j| < \varepsilon,
    $$
    where the typical value of the tolerance is $\varepsilon = 0.25 \sigma_{\hat{y}}$, i.e. it is \\given as $\pm \tfrac{1}{4}$ st.err.(estimated propensity scores).  \\ \medskip
    Variants of caliper-based matching: \\ \medskip
    \begin{itemize}
        \item 1-to-1 using NNs within caliper (typical application)
        \smallskip
        \item 1-to-1 using Mahalanobis within caliper
        \smallskip
        \item 1-to-$n$ using NNs/Mahalanobis within caliper
        \smallskip
        \item selection with / without replacement
        \item non-treated matches can be based on actually observed individuals or synthetic (combined) control group members, etc\dots
    \end{itemize}
\end{itemize}
\end{frame}
%---------------------------------
\begin{frame}{PSM-based analysis}
\begin{itemize}
    \item \textbf{Mahalanobis \& caliper:} For each $i$th participant, we search \\for the 1 (or $n$) nearest available matching non-participants, based on Mahalandobis metric, within the $i$-specific caliper. Mahalanobis distance (metric) is defined as
    $$
    d(i,j) = (\bm{u}_i - \bm{v}_j)^{\prime} 
    \bm{C}^{-1} (\bm{u}_i - \bm{v}_j),
    $$
    where $\bm{u}_i$ and $\bm{v}_j$ are vectors of selected matching variables for a given participant $i$ and for some $j$th non-participant (within caliper limit), and $\bm{C}$ is the sample variance-covariance matrix of the matching variables from the full set of non-participants. \\ \medskip Mahalanobis distances are calculated and evaluated for all $\{i,j\}$ pairs within the caliper for the $i$th participant.
\end{itemize}
\end{frame}
%---------------------------------
\begin{frame}{PSM-based analysis}
\begin{enumerate}
    \item[4] \textbf{Assessing quality of the matching} \\ \smallskip
\end{enumerate}
\begin{itemize}
    \item Evaluate common support in covariates (follows the same logic \\as for the common support in PS)
    \smallskip
    \item Visual inspection of covariates (treated vs untreated)
    \smallskip
    \item $t$-tests of difference-in-means for covariates (treated vs untreated)
    \smallskip
    \item Calculate and assess the average absolute standardized difference (standardized imbalance)
\end{itemize}    
\end{frame}
%---------------------------------
\begin{frame}{PSM-based analysis}
\begin{itemize}
    \item Visual inspection of covariates (treated vs untreated)\\
    \medskip
    Plot the mean of each covariate against the estimated propensity score, separately by treatment status. \\ \medskip If matching is ``done well'', the treatment and control groups will have (near) identical means of each covariate at each value of the propensity score (recall the large sample assumption for PSM).\\ \medskip
    \centering
    PSM -- Common support in covariates example \\ \medskip
\includegraphics[trim = 0.4cm 23cm 0cm 23cm, clip, width=0.7\textwidth]{./IMG/PSM_vis_insp.jpg} 
\end{itemize}
\end{frame}
%---------------------------------
\begin{frame}{PSM-based analysis}
\begin{itemize}
    \item Average absolute standardized difference (standardized imbalance) \\ \medskip
    A popular measure of of the standardized average imbalance is the following metric: \\ \medskip
    $$
    \textit{SAI}_k = 
    \frac{1}{k} \sum_x \frac{\vert \beta_x \vert }{\sigma_x},
    $$
    where $| \beta_x |$ is the absolute value of difference between the covariate $x$ means in the treated and control groups in the matched sample, $k$ is the number of covariates and $\sigma_x$ is the standard deviation of the covariate $x$ in the matched sample. \\ \medskip An average absolute standardized difference that is close to 0 is preferable, since that indicates small differences between the control and treatment groups in the matched sample.
\end{itemize}
\end{frame}
%---------------------------------
\begin{frame}{PSM-based analysis}
\begin{enumerate}
    \item[5] \textbf{Analyzing propensity-matched data} \\ \medskip
    Estimating TE is simple, once we have a matched sample that \\we consider well balanced. 
\end{enumerate}
\medskip
\begin{itemize}
    \item We use a $t$-test on means of the outcome variable (treated vs untreated)
    \smallskip
    \item We use OLS: SLRM with outcome variable regressed on the treatment variable
    \smallskip
    \item We use OLS, expand model by covariates
    \smallskip
    \item DiD on matched sample (OLS-based analysis)
    \smallskip
    \item Apply any model / analysis tool of choice to the matched data.
\end{itemize}    
\end{frame}
%---------------------------------
% R tutorial: 
% https://www.ncbi.nlm.nih.gov/pmc/articles/PMC8246231/
% https://simonejdemyr.com/r-tutorials/statistics/tutorial8.html
% https://www.publichealth.columbia.edu/research/population-health-methods/propensity-score-analysis 
% https://bookdown.org/cuborican/RE_STAT/propensity-score-match.html
% 
% https://journals.lww.com/anesthesia-analgesia/fulltext/2018/10000/five_steps_to_successfully_implement_and_evaluate.37.aspx
%---------------------------------
\begin{frame}{Treatment effects analysis}
\textbf{Estimation approaches for TE analysis:}\\
\bigskip
\begin{enumerate}
    \item \textcolor{lightgray}{Differences in differences (DiD)
    \bigskip
    \item Propensity score matching (PSM)}
    \bigskip
    \item Regression discontinuity design (RDD)
\end{enumerate}
\end{frame}
%---------------------------------
\section{Regression discontinuity design}
\begin{frame}{Regression discontinuity design}
\centering
RDD-based treatment effect visualization example \\(linear function \& common slopes assumed).

\begin{figure}
    \includegraphics[trim = 0cm 0cm 0cm 0.65cm, clip,width=0.7\textwidth]{./IMG/RDD1.jpg}
    \label{fig:my_label}
\end{figure}    
\end{frame}
%---------------------------------
\begin{frame}{RDD: Assumptions and features}
 \footnotesize  
\begin{itemize}
    \item \emph{Assignment Variable}: RDD requires a measurable assignment variable that determines whether an observation falls into the treatment or control group.
    \item \emph{Threshold}: specific value of the assignment variable that determines the assignment to treatment (deterministic and fuzzy variants of RDD exist).
    \item \emph{Local Randomization}: Around threshold, the assignment to treatment is as good as random, which allows for causal interpretation of TE.
    \item \emph{Continuity}: The expected outcome in the absence of treatment must be a continuous function of the assignment variable at the threshold.
    \item \emph{Bandwidth Selection}: Bandwidth around threshold needs to be carefully chosen to balance bias and variance.
    \bigskip
    \item RDD estimates are local to the threshold and may not generalize to the whole population -- unlike PSM, which gives ATET (PSM-based TEs relate to the treated and assume that TEs are constant across all individuals).
    \item RDD requires fewer data assumptions and can be used with fewer observations (compared to PSM) 
\end{itemize}
    
\end{frame}
%---------------------------------
\begin{frame}{RDD: description of main features}
\textbf{Deterministic / sharp RDD}\\ \smallskip
\begin{itemize}
    \item There is a continuous variable $X_i$ that determines who gets treated, ($D_i = 1$ if treated). 
    \item $X$ is called the running / assignment variable.
    \item In sharp RDD, a unit is treated if $X_i \geq c$ and not treated if $X_i < c$, with $c$ being a given threshold.
    \item We must observe $X$ and know the cutoff/threshold value $c$.
\end{itemize}
\bigskip
\textbf{Fuzzy RDD}\\ \smallskip
\begin{itemize}
    \item In fuzzy RDD, $D_i$ is a random variable, influenced by $X$.
    \item $E[D_i|X_i] = P[D_i=1|X_i]$ is discontinuous at $X_i = c$.
    \item $X$ with values close to $c$ (at $c$) is a predictor of treatment, but it does not completely determine treatment assignment.
\end{itemize}
\end{frame}
%---------------------------------
\begin{frame}{RDD: description of main features}
With sharp RDD, we can write the following limits \\(note the direction of arrows):\\ \medskip
\begin{itemize}
    \item $\lim_{x\rightarrow c}E[y_i|X_i =x, D_i=0] \approx E[y_{0i}|X_i = c],$
    \medskip
    \item $\lim_{x \leftarrow c}E[y_i|X_i =x, D_i=1] \approx E[y_{1i}|X_i = c],$
\end{itemize} \medskip
where $y_{0i}$ and $y_{1i}$ relate to the potential outcome with/without treatment and the difference between the two expected outcomes at $X_i=c$ is the treatment effect (TE evaluated at $X_i=c$). \\ \bigskip
In practical applications, RDD-based TE evaluation is calculated for a specific $X$-value range around $c$ (bandwidth selection).
\end{frame}
%---------------------------------
\begin{frame}{RDD: description of main features}
\textbf{RDD: Variance vs Bias}
\begin{itemize}
    \item TE estimated based on small subsample -- the value of the running variable is close to $c$. Smaller sample $\rightarrow$ larger st. errors.
    \item Select a larger sample and estimate parametrically (e.g. use regression model). Approach depends on  the functional form \\and polynomials used in model. 
    \item Choice of model specification is complicated by lack of overlap.
\end{itemize}
\bigskip
\textbf{Sharp RDD: Lack of overlap}
\begin{itemize}
    \item Overlap requires $0 < P(D_i = 1|X_i ) < 1$ for the domain of $X_i$
    \item Sharp RDD: $P(D_i = 1|X_i < c) = 0$ and $P(D_i = 1|X_i \geq c) = 1$
    \item We rely on extrapolation (counterfactuals) to estimate TE.
    \item TE can be wrong, if we use a wrong functional form $y_i = f(X_i,\bm{x}_i )$.
    \item We never know the actual $f(\cdot)$. Model specification is a key issue: use non/semi/parametric method.
\end{itemize}
\end{frame}
%---------------------------------
\begin{frame}{RDD: description of main features}
\centering
RDD-based treatment effect visualization example\\
(quadratic function \& separate slopes assumed).\\ \medskip
\begin{figure}
    \includegraphics[trim = 0cm 0cm 0cm 0cm, clip, width=0.6\textwidth]{./IMG/RDD2.png}
\end{figure}    
\end{frame}
%---------------------------------
\begin{frame}{RDD: estimation algorithm}
\begin{enumerate}
    \item Check the data and assumptions (sharp RDD)
    \medskip
    \item Estimate linear model, start with common slopes
    \medskip
    \item Include polynomials, allow for separate slopes
    \medskip
    \item Modify functional form: use semi/non parametric estimation, splines, LOESS (locally estimated scatterplot smoothing), etc.
    \medskip
    \item Sensitivity analysis (w.r.t. to sample size, etc.)
\end{enumerate}
\bigskip
With fuzzy RDD, TE estimation can follow the IV (instrumental variable) approach. We use the assignment/running variable as an IV. \textcolor{blue}{\underline{\href{https://clas.ucdenver.edu/marcelo-perraillon/teaching/health-services-research-methods-i-hsmp-7607}{(see lecture notes for RDD here)}}}
\end{frame}
%---------------------------------
\begin{frame}{RDD -- potential problems}
Misspecification of the model/functional form
\begin{itemize}
    \item Linearity is often the default approach in regression models. Can be addressed by using non-linear models.
    \item Possibly, once a nonlinear relationship is modelled correctly, discontinuity disappears as it was caused by a misspecification in the functional form. In the plot below, the correct relationship is specified by a curve. If we use a linear relationship, we might observe a (spurious) discontinuity.
\end{itemize}
\begin{figure}
    \centering
    \includegraphics[width=0.55\textwidth]{./IMG/RDD3.png}
\end{figure}    
\end{frame}
%---------------------------------
\begin{frame}{Treatment effects - additional literature}
\textbf{Treatment effects analysis}\\
\medskip
For detailed \& technical discussion of TEs, see:\\
\medskip
\begin{itemize}
\item[1.] Greene: Econometric analysis, chapter 19.6
\item[2.] Angrist, Pischke: Mostly Harmless Econometrics
\item[3.] Cameron, Trivendi: Microeconometrics, Methods and Applications, chapter 25
\item[4.] Wooldridge: Econometric analysis of C-S and panel data, chapter 21 Estimating Average Treatment Effects
\bigskip
\item \textcolor{blue}{\underline{\href{https://clas.ucdenver.edu/marcelo-perraillon/teaching/health-services-research-methods-i-hsmp-7607}{Health Services Research Methods I (HSMP 7607), UC Denver}}}
\item \textcolor{blue}{\underline{\href{https://simonejdemyr.com/r-tutorials/statistics/tutorial8.html}{https://simonejdemyr.com/r-tutorials/statistics}}}
\item \textcolor{blue}{\underline{\href{https://rpubs.com/sharmaar/RDD}{https://rpubs.com/sharmaar/RDD}}}
\end{itemize}
\end{frame}
%---------------------------------
\end{document}