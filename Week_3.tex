%%%%%%%%%%%%%%%%%%%%%%%%%%%%%%%%%%%%%%%%%
% Beamer Presentation
% LaTeX Template
% Version 1.0 (10/11/12)
%
% This template has been downloaded from:
% http://www.LaTeXTemplates.com
%
% License:
% CC BY-NC-SA 3.0 (http://creativecommons.org/licenses/by-nc-sa/3.0/)
%
%%%%%%%%%%%%%%%%%%%%%%%%%%%%%%%%%%%%%%%%%

%----------------------------------------------------------------------------------------
%	PACKAGES AND THEMES
%----------------------------------------------------------------------------------------

\documentclass{beamer}

\mode<presentation> {

\usetheme{Madrid}
\usefonttheme{serif} 
\setbeamertemplate{navigation symbols}{}
}

\usepackage{graphicx} % Allows including images
\usepackage{booktabs} % Allows the use of \toprule, \midrule and \bottomrule in tables
\usepackage[T1]{fontenc}
\usepackage[utf8]{inputenc}
\usepackage{amsmath}
\usepackage{color}
\usepackage[czech]{babel}
\usepackage{lmodern}  
\usepackage{rotating}
\usepackage{scrextend}
\usepackage{pifont}
\usepackage{hyperref}
\usepackage{bm}

%----------------------------------------------------------------------------------------
%	TITLE PAGE
%----------------------------------------------------------------------------------------

\title[Týden 3]{Praktikum z ekonometrie} % The short title appears at the bottom of every slide, the full title is only on the title page

\author{VŠE Praha} % Your name
\institute[4EK417] % Your institution as it will appear on the bottom of every slide, may be shorthand to save space
{
% Your institution for the title page
\medskip
\textit{Tomáš Formánek} % Your email address
}
\date{} % Date, can be changed to a custom date

\begin{document}

\begin{frame}
\titlepage % Print the title page as the first slide
\end{frame}

\begin{frame}
\frametitle{Outline} % Table of contents slide, comment this block out to remove it
\tableofcontents % Throughout your presentation, if you choose to use \section{} and \subsection{} commands, these will automatically be printed on this slide as an overview of your presentation
\end{frame}

%----------------------------------------------------------------------------------------
%	PRESENTATION SLIDES
%---------------------------------------------------------------------------------------
\section{The nature of missing data}
\begin{frame}
\frametitle{The nature of missing data}

\begin{itemize}
  \item[] Missing completely at random (\textit{MCAR})
  \begin{itemize}
  \item The probability that an observation $X_i$ is missing is unrelated to the value of $X_i$ or to the value of any other variables.
  \item Any piece of data is equally likely to be missing.
  \item Analyses based on data with \textit{MCAR} observations remain unbiased. We may lose power (increased standard errors), but the estimated parameters are not biased by the absence of data.
  \end{itemize}
  \vspace{0.2cm}
  \item[] Missing at random (\textit{MAR})
  \begin{itemize}
  \item Data meets the requirement that missingness does not depend on the value of $X_i$ after controlling for another variable in our analysis.
  \item For example, data are MCAR in a specific (demographic) subgroup. 
  \end{itemize}
  \vspace{0.2cm}
  \item[] Missing Not at Random (\textit{MNAR})
  \begin{itemize}
  \item Missigness of $X_i$ depends on its value (e.g. income in surveys)
  \item The only way to obtain an unbiased estimates of (regression) parameters is to model the missingness.
  \end{itemize}
\end{itemize}


\end{frame}

%------------------------------------------------
\section{Traditional treatment of missing data}
\begin{frame}[fragile] % Need to use the fragile option when verbatim is used in the slide
\frametitle{Traditional treatment of missing data}


\textbf{Listwise deletion (complete cases analysis)}
\vspace{0.2cm}
  \begin{itemize}
  \item  We omit all rows with missing data – missing information for at least one variable in the $i$-th individual observation. Then, we run our analyses on the observations that remain. This often results in a substantial decrease in sample size. Under the assumption that data are missing completely at random, LRMs lead to unbiased parameter estimates – still, we lose power due to exclusion of (potentially large number of) observations.
  \end{itemize} 

\begin{block}{R code}
\begin{verbatim}
newData <- data[complete.cases(data)==T, ] 
# data is a data.frame
# or
newData <- na.omit(data) 
\end{verbatim}
\end{block}

\end{frame}

%------------------------------------------------

\begin{frame}
\frametitle{Traditional treatment of missing data}


\textbf{Hot deck imputation}
\vspace{0.2cm}
  \begin{itemize}
  \item  Historically used by the US Census Bureau (since 1950’s). Respondent’s missing data were replaced by observed replacement data – drawn at random from a group of similar participants. Suitable, given only a few missing observations need to be replaced and given the draw is random.
  \end{itemize} 

\end{frame}


%------------------------------------------------


\begin{frame}
\frametitle{Traditional treatment of missing data}


\textbf{Mean substitution}
\vspace{0.2cm}
  \begin{itemize}
  \item[\ding{51}] Simple
  \item[\ding{55}] In simple linear regression models (SLRMs), this adds no new information but increases sample size – that leads to underestimated standard errors only.
  \end{itemize} 
  \vspace{0.2cm}
\textbf{Example:} Data on salary and citation level of publications. 62 cases with complete data and 7 cases for which the citation index was missing. Correlations and regression coefficients were compared as follows:

\begin{table}
\begin{tabular}{l c c c c}
\toprule
Analysis & $n$ & $corr$ & $\widehat{\beta}_1$ & $\textit{s.e.}(\widehat{\beta}_1)$\\
\midrule
Complete cases only & 62 & .55 &  310.747 & 60.95 \\
With mean substitution & 69 & .54 &  310.747 & 59.12 \\
\bottomrule
\end{tabular}

\end{table}

\end{frame}

%------------------------------------------------

\begin{frame}
\frametitle{Traditional treatment of missing data}

\textbf{Regression substitution}
  \begin{itemize}
  \item Uses linear regression (auxiliary LRM) to predict what the missing values of regressors should be, on the basis of other variables that are present.
  \item For SLRMs, the same problem of error variance as in mean substitution remains. We do not add more information but we increase the sample size and (spuriously) reduce the standard error.
  \item May be useful for MLRMs.
   \end{itemize}
  \vspace{0.2cm}
\textbf{Stochastic regression substitution}
  \begin{itemize}
  \item This approach adds a randomly sampled residual term from the normal (or other) distribution to each value estimated by regression substitution. Adding a bit of random error to each substitution reduces, but does not eliminate, the problem of spurious reduction of the standard errors.
   \end{itemize}

\end{frame}


%------------------------------------------------
\section{Modern Approaches to missing data}
\begin{frame}
\frametitle{Modern Approaches to missing data}

\textbf{Maximum Likelihood Expectation-Maximization}
  \begin{itemize}
  \item Computationally complex, maximum likelihood approach to the estimation of missing values Many approaches exist (e.g. the Expectation-Maximization algorithm)
  \item[] \scriptsize{\url{https://www.uvm.edu/~dhowell/StatPages/Missing_Data/Missing-Part-Two.html}}
   \end{itemize}
 
\end{frame}

%------------------------------------------------



\begin{frame}
\frametitle{Modern Approaches to missing data}

\textbf{Multiple Imputation (MI) }
  \begin{itemize}
  \item[] R: $\left\lbrace \textnormal{mice}  \right\rbrace , \,\, \left\lbrace \textnormal{mi}  \right\rbrace, \,\, \left\lbrace \textnormal{Amelia}  \right\rbrace, \, $ \dots
  \vspace{0.5cm}
  \item[] MI motivation and algorithm
  \vspace{0.2cm}
  \begin{itemize}
  \item Create several (say, 5) “imputed” values for each missing value $X_{ij}$. Each of the (5) versions of imputed data values are estimated/predicted using a separate ML approach from the data frame observed (different “model” is used for each imputation).
  \vspace{0.2cm}
  \item For SLRMs, this may be simplified into mean substitution augmented by adding random errors which reflect sampling variability of $X_j$.
  \end{itemize}
 \end{itemize}
 
\end{frame}


%------------------------------------------------


\begin{frame}
\frametitle{Modern Approaches to missing data}
\vspace{0.2cm}
\textbf{Multiple Imputation (contnd.) }
\vspace{0.2cm}
  \begin{itemize}
  \item[] How do we analyze estimates on data with MI?
  \begin{enumerate}
  \vspace{0.5cm}
  \item We use each set of imputed values to form a separate completed dataset (e.g., we get 5 datasets).
  \vspace{0.2cm}
  \item For each completed dataset, a standard analysis (LRM) can be run.
  \vspace{0.2cm}
  \item Inferences can be combined across MI-based datasets.
  \end{enumerate}
 \end{itemize}
 
\end{frame}

%------------------------------------------------


\begin{frame}

\frametitle{Modern Approaches to missing data}
\textbf{Multiple Imputation (contnd.) }
\begin{figure}
\includegraphics[width=0.7\linewidth]{IMG/mitable.jpg}
\end{figure}

\scriptsize{\url{https://www.uvm.edu/~dhowell/StatPages/Missing_Data/Missing-Part-Two.html}}

\end{frame}


%------------------------------------------------
\section{Missing dependent variable data}
\begin{frame}
\frametitle{Missing dependent variable data}

\textbf{Special considerations apply to missing dependent variable data}

\begin{itemize}
  \item If we can assume that data are missing completely at random \textit{(MCAR)}, we will lose power because of smaller sample sizes, but we will not have problems with biased estimates.
  \item If data are missing not at random \textit{(MNAR)}, the \textbf{only way to obtain an unbiased estimate of parameters is to model missingness}. In other words we need to use a model that accounts for the missing data.
  \item Broadly speaking, such models are:
  \begin{itemize}
    \item Censored Regression Models (e.g. duration analysis) 
    \item Truncated Regression Models 
    \item Sample Selection Correction models (Heckit)
    \item \dots
  \end{itemize}
\end{itemize}

\end{frame}


%------------------------------------------------



























%---------------------------------------------------------------------

\end{document}