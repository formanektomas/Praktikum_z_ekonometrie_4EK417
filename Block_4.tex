%%%%%%%%%%%%%%%%%%%%%%%%%%%%%%%%%%%%%%%%%
% Beamer Presentation
% LaTeX Template
% Version 1.0 (10/11/12)
%
% This template has been downloaded from:
% http://www.LaTeXTemplates.com
%
% License:
% CC BY-NC-SA 3.0 (http://creativecommons.org/licenses/by-nc-sa/3.0/)
%
%%%%%%%%%%%%%%%%%%%%%%%%%%%%%%%%%%%%%%%%%

%----------------------------------------------------------------------------------------
%	PACKAGES AND THEMES
%----------------------------------------------------------------------------------------

\documentclass{beamer}

\mode<presentation> {
\usetheme{Madrid}
\usefonttheme{serif} 
\setbeamertemplate{navigation symbols}{} % To remove the navigation symbols from the 
}

\usepackage{graphicx} % Allows including images
\usepackage{booktabs} % Allows the use of \toprule, \midrule and \bottomrule in tables
\usepackage[T1]{fontenc}
\usepackage[utf8]{inputenc}
\usepackage{amsmath}
\usepackage{color}
\usepackage[czech]{babel}
\usepackage{lmodern}  
\usepackage{rotating}
\usepackage{scrextend}
\usepackage{pifont}
\usepackage{hyperref}
\usepackage{bm}
%
\newcommand{\Rko}{\texttt{R~}} 
%----------------------------------------------------------------------------------------
%	TITLE PAGE
%----------------------------------------------------------------------------------------
\title[Block 4]{Praktikum z ekonometrie} % The short title appears at the bottom of every slide, the full title is only on the title page
\author{VŠE Praha} % Your name
\institute[4EK417] % Your institution as it will appear on the bottom of every slide, may be shorthand to save space
{
% Your institution for the title page
\medskip
\textit{Tomáš Formánek} % Your email address
}
\date{} % Date, can be changed to a custom date
%----------------------------------------------------------------------------------------
\begin{document}
\begin{frame}
\titlepage % Print the title page as the first slide
\end{frame}
%---------------------------------------------------------------------
\begin{frame}
\frametitle{Block 4 – Linear mixed effect models – Outline
} % Table of contents slide, comment this block out to remove it
\tableofcontents % Throughout your presentation, if you choose to use \section{} and \subsection{} commands, these will automatically be printed on this slide as an overview of your presentation
\end{frame}

%---------------------------------------------------------------------
%	PRESENTATION SLIDES
%---------------------------------------------------------------------
\section{Introduction}
\begin{frame}{Introduction}
LME generalization of a linear regression model:
\medskip
$$
    \bm{y} = \bm{X \beta} + \bm{Z u} + \bm{\omega},  
    \qquad \bm{u} \sim N(\bm{0}, \bm{G}),
    \qquad \bm{\omega} \sim N(\bm{0}, \bm{R}),
$$
\medskip
\small
where 
\begin{itemize}
    \item[] $\bm{y}$ is a vector of dependent variable observations,
    \item[] $\bm{X}$ is a $(N\! \times \! p)$ matrix of ``fixed effects'' with $N$ observations and $p$ predictor variables,
    \item[] $\bm{\beta}$ is a vector of ``fixed-effects'' regression coefficients,
    \item[] $\bm{Z}$ is a $(N\! \times \! s)$ design matrix for the $s$ ``random effects'' $\bm{u}$ that are complementary to $\bm{\beta}$,
    \item[] $\bm{\omega}$ is the error term,
    \item[] $\bm{u}$ and $\bm{\omega}$ are assumed normally distributed and mutually independent, with variance-covariance matrices $\bm{G}$ and $\bm{R}$ respectively.
\end{itemize}   
\end{frame}
%---------------------------------------------------------------------
\begin{frame}{Introduction}
\begin{itemize}
    \item \textbf{Fixed effects:} variables with expected impacts (effects) on the dependent/response variable.\\ \medskip
    Essentially the same as explanatory variables in a standard linear regression models.
    \bigskip
    \item \textbf{Random effects:} typically, grouping factors that we are trying to control for. \\ \medskip RE are always categorical variables (\texttt{R} cannot treat continuous variables as random effects). \\ \medskip 
    Often, we are not interested in specific impacts of individual random effects on the response variable -- but we know RE might be influencing the patterns we see \& we want to control for such processes.
\end{itemize}
\end{frame}
%---------------------------------------------------------------------
\begin{frame}{Introduction}

LME generalization of a linear regression model -- example: 
$$y_i = \beta_0 + \beta_1 x_i + \omega_i.$$ 
If observations $i=1,\dots,N$ are organized into $j=1,\dots,J$ relevant groups, and assuming there are random effects on both $\beta$-coefficients, we may generalize the model into a LME (after re-arranging):
$$y_{ij} = (\beta_0 + u_{0j}) + (\beta_1 + u_{1j}) x_{ij} + \omega_{ij},$$
where the combined $ij$ subscript refers to an $i$th observation that belongs to a $j$th group. 

\medskip
In the LME model, $u_{0j}$ and $u_{1j}$ are stochastic deviations from $\beta$-coefficients that are associated with a particular $j$th group. While random effects may look like model coefficients, we are only interested in estimating their variances. 
\end{frame}
%---------------------------------------------------------------------
\begin{frame}{Introduction}
\small 
\textbf{Linear mixed effect model (LME)} -- data types:
\medskip
\begin{itemize}
\item \textbf{Longitudinal data:} repeated measurements are performed on each individual  unit. Several units are sampled. Number of observations may differ across units.\\ \smallskip
\quad $y_{ti}$ - observation at time $t$ for $i$-th individual. \\  
\quad $y_{ij}$ - $i$th observation of $j$th individual (if time aspect not relevant). \\ 
\bigskip
\item \textbf{Hierarchical data structures:} data with two or more groups/levels of observations. Number of observations may differ across units.\\ \smallskip
\quad $y_{ij}$ - observation for $i$-th company within $j$-th region. \\ 
\quad $y_{ij}$ - observation for $i$-th student within $j$-th class. \\ 
\medskip
\bigskip
\item \textbf{Combined:} We can group observations at three levels (or more):\\
\smallskip
\quad $y_{tij}$ - measurement at time period $t$, admin. region $i$ within state $j$.\\ 
\bigskip
\item Note how indices are ordered (left to right) from individual to highest level of aggregation. (alternative orderings exist in literature).
\end{itemize}
\end{frame}
%---------------------------------------------------------------------
\begin{frame}{Longitudinal data}
\begin{itemize}
\item $N$ individual CS units are followed over time.
\medskip
\item The observation set $\{y_{ti},\bm{x}_{ti}\}$ denotes some $i$th individual  observed at a time period $t$. The number of observations in time may differ among CS units and observations may occur at different time points. \\ \medskip 
\textbf{Example:} For a medical study, we measure child's weight (plus other data) at birth and repeatedly over a period of one year. For some $y_{ti}$ observation, index $t$ denotes days from birth. Due to doctor visit scheduling, children are weighted at different $t$ ``values''. Typically, the number of doctor visits (observations) differs across children. Children in the study are born on different dates (say, Jan 2015 - Oct 2019). \\ \medskip Example extends easily to economic environment \\(we can follow newly founded companies, etc.).
\end{itemize}
\end{frame}
%---------------------------------------------------------------------
\begin{frame}{Hierarchical data structures}
\begin{itemize}
\item Nested/hierarchical structure of the LME model:
\medskip
\begin{itemize}
    \item Individual units $i$ (Level 1) are nested
    \medskip
    \item within $j$ groups (Level 2) with group-specific observation sizes $n_j$.
\end{itemize}
\medskip
\item One or more coefficient(s) can vary across groups \\(``random effects).
\end{itemize}
\bigskip
\textbf{LME: Longitudinal vs. hierarchical data structures:}
\medskip
\begin{itemize}
\item Essentially, the same nesting/hierarchical framework applies to longitudinal data and their LME-based analysis:
\medskip
\begin{itemize}
    \item Observations at time $t$ (Level 1) are nested
    \smallskip
    \item within $j$ individual units (Level 2).
    \smallskip
    \item If appropriate, individual units can be nested in groups (Level 3) \dots 
\end{itemize}
\end{itemize}
\end{frame}
%---------------------------------------------------------------------
\begin{frame}{Introduction}
\begin{itemize}
\item Mixed models are called ``mixed'', because the $\beta$-coefficients are \\a mix of fixed parameters and random variables
\smallskip 
\item Terms ``fixed'' and ``random'' have specific meaning for LMEs:\\ \smallskip
\begin{itemize}
\item A fixed coefficient is an unknown constant to be estimated. \smallskip
\item A random coefficient varies from ``group'' to ``group''. \\By ``group'', we mean Level 2 aggregation, if data have 2 levels. \\ - coefficients vary among schools (Level 2), not within school.\\
- coeffs. vary across individuals (Level 2), not over time (Level 1).
\end{itemize}
\smallskip
\item  LME models can have some added complexity:
\smallskip
\begin{itemize}
\item Multiple levels of nesting
\smallskip
\item Crossed random effects
\smallskip
\item Correlations between different random coefficients.


\end{itemize}
\smallskip
\item Random  coefficients are not estimated, but they can be predicted.
\end{itemize}
\end{frame}
%---------------------------------------------------------------------
\section{LME model}
\begin{frame}{LME model example}
\begin{itemize}
\item Data: \\London Education Authority Junior School Project
dataset, \\- we have 887 students ($i$) in 48 different schools ($j$),\\- we want to predict 5th-year math scores.
\medskip
\item We may start by ignoring the school grouping and any possible regressors -- we have a trivial model (\textit{single-mean} model):
$$ \texttt{math5}_{ij} = \beta_0~+~\varepsilon_{ij}, \quad i=1,\dots,n_j,  \quad j=1,\dots, M,  \quad \varepsilon_{ij} \sim N(0,\sigma^2_{\varepsilon}) $$

where $M=48$ and $n_j$ differ among schools, $\texttt{math5}_{ij}$ is the observed  math score of $i$-th student at school $j$,
$\beta_0$ is the mean math score across our population (being sampled) and
$\varepsilon_{ij}$ is the individual deviation from overall mean.\\
\medskip
Population mean math score \& the variance of $\varepsilon$ are estimated by taking their sample counterparts. Any ``school effect'' is ignored.\\
\end{itemize}
\end{frame}
%---------------------------------------------------------------------
\begin{frame}{LME model}
\begin{itemize}
\item The school effect (differences among schools) may be incorporated in the model by allowing the mean of each school to be represented by a separate parameter (\textit{fixed effect})
$$ \texttt{math5}_{ij} = \beta_{0j}~+~\varepsilon_{ij}, \quad i=1,\dots,n_j,  \quad j=1,\dots, M,  \quad \varepsilon_{ij} \sim N(0,\sigma^2_{\varepsilon}) $$

where $\beta_{0j}$ is the school-specific mean math score and
$\varepsilon_{ij}$ is the individual deviation from the school-specific mean.\\
\smallskip
\begin{itemize}
\item \Rko syntax: \texttt{lm(math5 $\sim$ School-1, data=...)}\\
$\Rightarrow~M=48$ school-specific intercepts are estimated.\\
\medskip
\item Using the terminology of LME, $\beta_{0j}$ are fixed. Hence:\\
\begin{itemize}
    \item \textbf{Estimated intercepts only model (refer to) the specific sample}  of schools, while -usually- the main interest is in the population from which the sample was drawn. \smallskip
    \item OLS regression does not provide an estimate of the between-school variability, which is also of central interest.
\end{itemize}
\end{itemize}
\end{itemize}
\end{frame}
%---------------------------------------------------------------------
\begin{frame}{LME model}
\small
\begin{itemize}
\item \textit{Random effects} approach: LME model can solve the above problem by treating school effects as random variations around population mean.
\medskip
\item Ordinary model (with \textit{fixed effects})  can be reparametrized as: \\
\smallskip
\textcolor{blue}{$ y_{ij} = \beta_{0j} + \varepsilon_{ij}$} \\
\smallskip
$y_{ij} = \textcolor{orange}{\beta_0} + ( \beta_{0j} \textcolor{orange}{-\beta_0}) + \varepsilon_{ij}$,\\
\medskip
\textit{Random effect}: ~~$u_{0j} = \beta_{0j} - \beta_0$~~ is the school-specific deviation from overall mean $\beta_0$. It can be used to replace the the \textit{fixed effect} $\beta_{0j}\,$:\\
\medskip
$u_{0j} = \beta_{0j} - \beta_0 ~~~ \Rightarrow ~~~ \beta_{0j} = \beta_0 + u_{0j}.$ Hence:\\  \smallskip
\textcolor{blue}{$ y_{ij} = \beta_{0} + u_{0j} + \varepsilon_{ij}.$}\\
\medskip
\item $u_{0j}$ is a random variable, specific for the $j$-th school, with zero
mean and unknown variance $\sigma^2_u$. \\ \smallskip
$u_{0j}$ is a \textit{random effect}, associated with the particular sample units (schools are selected at random from the population).\\ 
\end{itemize}
\end{frame}
%---------------------------------------------------------------------
\begin{frame}{LME model}
\begin{itemize}
\item LME model with \textit{random effects} (on the intercept) is given as:
$$ y_{ij} = \beta_{0} + u_{0j} + \varepsilon_{ij}, \qquad u_{0j} \sim N(0,\sigma^2_u), \qquad \varepsilon_{ij} \sim N(0,\sigma^2_{\varepsilon}), $$
and we assume $u_{0j}$ are $iid$ and independent from $\varepsilon_{ij}$.\\
\smallskip
\begin{itemize}
\item Observations within the same school share the same random effect $u_{0j}$, hence are positively ``correlated'' with \textnormal{ICC} = $\sigma^2_u \, / (\sigma^2_u + \sigma^2_{\varepsilon})$ \\ (see ICC discussion on a separate slide).
\medskip
\item This \textit{random effects} model has three parameters: $\beta_{0},~ \sigma^2_u$ and $\sigma^2_{\varepsilon}$. (regardless of $M$, the number of schools).
\medskip
\item Note that the \textit{random effect} $u_{0j}$ ``looks like'' a coefficient, but we are only interested in estimating $\sigma^2_u$.
\medskip
\item However, upon observed data (and estimated model), we do make predictions using fitted values of $\hat{u}_j$.
\end{itemize}
\end{itemize}
\end{frame}
%---------------------------------------------------------------------
\begin{frame}{LME model}
\small
\begin{itemize}
\item Exogenous regressors can be used in LMEs (like in LRMs).\\
\smallskip
For example, \texttt{math5} grades depend on \texttt{math3} ($3{^{\textnormal{rd}}}$ year grades).\\
\medskip
$ \texttt{math5}_{ij} = \left( \beta_{0} + u_{0j} \right) + \beta_1 \, \texttt{math3}_{ij} + \varepsilon_{ij} $\\
\bigskip
\textcolor{orange}{alternative notation:} \\ \medskip
$ \texttt{math5}_{ij} = \beta_{0} + \beta_1 \, \texttt{math3}_{ij} + u_{0j} + \varepsilon_{ij} $\\ \bigskip
\textcolor{orange}{alternative notation:} \\ \medskip
\textit{Level 1}:~~~$ \texttt{math5}_{ij} = \beta_{0j} + \beta_1 \, \texttt{math3}_{ij} + \varepsilon_{ij} $\\ \smallskip
\textit{Level 2}:~~~~~~~~~$ \beta_{0j} = \beta_{0} + u_{0j} $\\
\bigskip
\begin{itemize}
\item Intercept has a random effect, given the $u_{0j}$ element.
\smallskip
\item Slope of the regression line for each school is fixed at $\beta_1$.\\
\dots \texttt{math3} has a \textit{fixed effect}.
\end{itemize}
\end{itemize}
\end{frame}
%---------------------------------------------------------------------
\begin{frame}{LME model: ICC}
\small
\begin{itemize}
\item \textbf{ICC:} Intra class correlation in a LME regression model: \qquad
$$ \textnormal{ICC}=\frac{\sigma^2_u}{\sigma^2_u + \sigma^2_{\varepsilon}} $$
\begin{itemize}
    \item Describes how strongly units in the same group are ``correlated''.
    \medskip
    \item While interpreted as a type of correlation, ICC operates on groups, rather than paired observations. 
    \medskip
    \item See \textcolor{blue}{\underline{\href{https://en.wikipedia.org/wiki/Intraclass_correlation}{link}}} for relation between ICC and actual correlation.
\end{itemize}
\end{itemize}
\end{frame}
%---------------------------------------------------------------------
\begin{frame}{LME model: ICC}
\small
\begin{itemize}
\item \textbf{ICC:} Intra class correlation in a LME regression model: \qquad
$$ \textnormal{ICC}=\frac{\sigma^2_u}{\sigma^2_u + \sigma^2_{\varepsilon}} $$
\medskip
\item Example: \qquad $\texttt{math5}_{ij} = \beta_{0} + \beta_1 \, \texttt{math3}_{ij} + u_{0j} + \varepsilon_{ij},$\\ \medskip
\qquad \qquad \qquad ~where $\sigma^2_u = \textnormal{var}(u_{0j})$ and $\sigma^2_{\varepsilon} = \textnormal{var}(\varepsilon_{ij})$.\\ \smallskip
\medskip
\item $\textnormal{ICC}$: ``correlation'' between \texttt{math5}  observations (randomly chosen) within a given school.
\bigskip 
\item ICC has another useful interpretation: Say, ICC = 0.6 in our $\texttt{math5}_{ij}$ example. Hence, differences between schools explain 60\% of ``remaining'' variance -- i.e. after the variance explained by fixed effects (i.e. by $\texttt{math3}_{ij}$) is subtracted.
\end{itemize}
\end{frame}
%---------------------------------------------------------------------
\begin{frame}{LME model: random effects on intercept and slope}
\small
\begin{itemize}
\item If teaching is different from school to school, it would make
sense to have different slopes for each of the schools.\\
\medskip
Instead of using \textit{fixed effects} on slopes (interaction terms \texttt{math3:School}), we use random slopes: \textcolor{blue}{$ u_{1j} = \beta_{1j} - \beta_1 $}. \\
\bigskip
$ \texttt{math5}_{ij} = \left( \beta_{0} + u_{0j} \right) + \left( \beta_1\, \texttt{math3}_{ij}  + u_{1j} \, \texttt{math3}_{ij}  \right) + \varepsilon_{ij}, $ \\
\bigskip
\textcolor{orange}{alternative notation:}\\
\medskip
$\texttt{math5}_{ij} = \underbrace{\beta_{0} + \beta_1 \,  \texttt{math3}_{ij}}_{\textit{fixed}}
+ \underbrace{u_{0j} + u_{1j} \, \texttt{math3}_{ij}}_{\textit{random}} 
+ \varepsilon_{ij}, $ \\
\bigskip
\begin{itemize}
\item We can test whether this extra complexity is justified.
\smallskip
\item ${u}_{0j}$ and ${u}_{1j}$ are often correlated, their independence can be tested.
\smallskip
\item Fitted values of $\texttt{math5}_{ij}$ can be produced, along with $\hat{u}_{0j}$ and $\hat{u}_{1j}$.
\end{itemize}
\end{itemize}
\end{frame}
%---------------------------------------------------------------------
\begin{frame}{LME: matrix form re-visited}
\begin{itemize}
\item LME model:
$$\bm{y} = \bm{X \beta} + \bm{Z u} + \bm{\varepsilon} 
\qquad \bm{u} \sim N(\bm{0}, \bm{G})
\qquad \bm{\varepsilon} \sim N(\bm{0}, \bm{R}),$$
where:\\
$\bm{X}$ is a $(n \! \times \! k)$ matrix, $k$ is the number of \textit{fixed effects},\\
$\bm{Z}\,$ is a $(n \! \times \! p)$ matrix, $p$ is the number of \textit{random effects},\\
$\bm{G}\,$ is a $(p \! \times \! p)$ variance-covariance matrix of the \textit{random effects},\\
$\bm{R}\,$ is a $(n \! \times \! n)$ variance-covariance matrix of errors.\\
\medskip
\begin{itemize}
    \item Independence between $\bm{u}$ and $\bm{\varepsilon}$ is assumed,\\
    \item  Often, $\bm{R}=\sigma^2_{\varepsilon} \bm{I}_n$ is assumed,\\
    \item $\bm{G}\,$ is diagonal if \textit{random effects} are mutually independent.
\end{itemize}
\medskip
\item Estimation: MLE, RMLE, penalized least squares
\small
\\ \url{https://www.jstatsoft.org/article/view/v067i01/0}\\
\end{itemize}
\end{frame}
%---------------------------------------------------------------------
\begin{frame}{LME: R syntax}
\small 
For a single random effect (on the intercept): \\
\bigskip
\begin{itemize}
    \item \texttt{\{nlme\}} package: \\ \medskip
    \texttt{lme( y $\sim$ x + z, random = $\sim$ 1 | g, data = df )}
    \bigskip
    \item \texttt{\{lme4\}} package: \\ \medskip
    \texttt{lmer( y $\sim$ x + z + ( 1 | g ), data = df )}
    \bigskip
    \item where \texttt{y} is the response variable with predictors \texttt{x} and \texttt{z}, \\and grouping factor variable \texttt{g}.
\end{itemize}
\end{frame}
%---------------------------------------------------------------------
\begin{frame}{LME: R syntax}
\small 
For random effects on the intercept and \texttt{x}: \\
\bigskip
\begin{itemize}
    \item \texttt{\{nlme\}} package: \\ \medskip
    \texttt{lme( y $\sim$ x + z, random = $\sim$ 1 + x | g, data = df )}\\
    \texttt{lme( y $\sim$ x + z, random =~~~~~$\sim$ x | g, data = df )}
    \bigskip
    \item \texttt{\{lme4\}} package: \\ \medskip
    \texttt{lmer( y $\sim$ x + z + ( 1 + x | g ), data = df )} \\
    \texttt{lmer( y $\sim$ x + z + ~~~ ( x | g ), data = df )}
\end{itemize}
\end{frame}
%---------------------------------------------------------------------
\begin{frame}{LME: R syntax}
\small 
For uncorrelated random effects on the intercept and \texttt{x}: \\
\bigskip
\begin{itemize}
    \item \texttt{\{nlme\}} package: \\ \medskip
    \texttt{lme( y $\sim$ x + z, random = list ( g = pdDiag ( $\sim$ x ) ) , \\~~~~~data = df )}
    \bigskip
    \item \texttt{\{lme4\}} package: \\ \medskip
    \texttt{lmer( y $\sim$ x + z + ( 1 | g ) + (0 + x | g) , data = df )} \\
    \texttt{lmer( y $\sim$ x + z + ( x || g ) , data = df )}
\end{itemize}
\end{frame}
%---------------------------------------------------------------------
\section{Complex LME models}
\begin{frame}{Complex LME models - brief outline}
Different types of LME models exist:\\
\bigskip
\begin{itemize}
\item LME models with (multilevel) nested effects,
\medskip
\item LME models with crossed effects,
\medskip
\item Complex behavior of the error term in LME models can be addressed (heteroscedasticity and serial correlation).
\medskip
\item LME models with non-Gaussian dependent variables \\(binary, Poisson, etc.).
\end{itemize}
\end{frame}
%---------------------------------------------------------------------
\begin{frame}{LME models with multilevel nested effects}
\textbf{Multi-level model example:} For 17 years, we follow a total of 86 individual states organized within 9 ``global-level'' regions (e.g. South America, Europe, Middle East, etc.).
\medskip
\begin{itemize}
\item $\texttt{GDP}_{tij}$ represents individual GDP per capita measurements for:\\
\smallskip
$t$-th time period, e.g. with values ($t= 2000, \dots, 2016)$.\\
$i$-th state nested within region $j$ ($i=1,\dots,M_j$),\\
$j$-th region ($j=1,\dots,9$),\\
\medskip
\item We fit \texttt{GDP} as a function of productivity \texttt{P} and unemployment \texttt{U}.\\
\smallskip
States are nested in regions, we have 2 levels of random intercepts: \\
~ $u_{0i(j)}$ for each state (within a region),\\
~ $v_{0j}$ for the regions,\\
~ random slopes can be added as well.
\bigskip
\item $\texttt{GDP}_{tij} = \beta_0 + \beta_1 \, \texttt{P}_{tij} 
+ \beta_2 \, \texttt{U}_{tij} + u_{0i(j)} + v_{0j}  + \varepsilon_{tij}.$
\end{itemize}
\end{frame}
%---------------------------------------------------------------------
\begin{frame}{LME: R syntax}
\small 
For intercept varying among \texttt{g1} and \texttt{g2} within \texttt{g1}: \\
\textcolor{blue}{For intercept \& \texttt{x} varying among \texttt{g1} and \texttt{g2} within \texttt{g1}:}
\bigskip
\begin{itemize}
    \item \texttt{\{nlme\}} package: \\ \medskip
    \texttt{lme( y $\sim$ x + z, random = $\sim$ 1 |  g 1 / g 2 , data = df )}\\
    \textcolor{blue}{\texttt{lme( y $\sim$ x + z, random = $\sim$ x |  g 1 / g 2 , data = df )}}
    \bigskip
    \item \texttt{\{lme4\}} package: \\ \medskip
    \texttt{lmer( y $\sim$ x + z +( 1 | g 1 / g 2 ), data = df )}\\
    \textcolor{blue}{\texttt{lmer( y $\sim$ x + z +( x | g 1 / g 2 ), data = df )}}
\end{itemize}
\end{frame}
%---------------------------------------------------------------------
\begin{frame}{LME models with crossed random effects}
Crossed \textit{random effects} example:\\
\medskip
\begin{itemize}
\item Grunfeld (1958) analyzed data on 10 large U.S. corporations,
collected annually from 1935 to 1954 to investigate how \\investment \texttt{I} depends on market value \texttt{M} and capital stock \texttt{C}.\\
\medskip
\item Here, we want \textit{random effects} for a given firm and year.\\
\smallskip
We want the year effect to be the same across all firms, \\i.e. not nested within firms.
\bigskip
\item $\texttt{I}_{ti} = \beta_0 + \beta_1 \, \texttt{M}_{ti} 
+ \beta_2 \, \texttt{C}_{ti} + u_{0i} + v_{0t} + \varepsilon_{ti}.$\\
\medskip
where $i=1, \dots, 10$ and \\firms are followed over $t=1,\dots,20$ years.
\\(the usual ``$it$'' index ordering can be used as well)
\end{itemize}
\end{frame}
%---------------------------------------------------------------------
\begin{frame}{LME: R syntax}
\small 
For intercept varying among \texttt{g1} and \texttt{g2} \\
\bigskip
\begin{itemize}
    \bigskip
    \item \texttt{\{lme4\}} package: \\ \medskip
    \texttt{lmer( y $\sim$ x + z + ( 1 | g 1 ) + ( 1 | g 2 ), data = df )}\\
\end{itemize}
\end{frame}
%---------------------------------------------------------------------
\begin{frame}{LME with heteroscedasticity and serial correlation}
\small 
\texttt{\{nlme\}} package: \\
\bigskip
\begin{itemize}
    \item Heteroscedastic residual variance at level 1:\\ \medskip
    \texttt{lme( y $\sim$ x + z, random = $\sim$ 1 |  g , \\
           ~~~~~~~~~weights = varIdent ( form = $\sim$ 1 | g ) , data  )}\\
    \bigskip
    \item Autoregressive \texttt{ar(1)} residuals:\\ \medskip
    \texttt{lme( y $\sim$ x + z, random = $\sim$ 1 |  g , \\
           ~~~~~~~~~correlation = corAR1 ( form = $\sim$ time ) , data  )}\\
    \bigskip
    \item General residuals:\\ \medskip
    \texttt{lme( y $\sim$ x + z, random = $\sim$ 1 |  g , \\
           ~~~~~~~~~weights = varIdent ( form = $\sim$ 1 | g ) , \\
           ~~~~~~~~~correlation = corAR1 ( form = $\sim$ time ) , data )}\\
\end{itemize}
\end{frame}
%---------------------------------------------------------------------
\section{LME models -- references, R packages}
\small
\begin{frame}{LME models -- references, R packages}
\begin{itemize}
    \item \texttt{\{lme4\}} package \\ \smallskip
    \url{https://www.jstatsoft.org/article/view/v067i01/0}\\
    \bigskip 
    \item \texttt{\{nlme\}} package \\ \smallskip
    \url{https://cran.r-project.org/web/packages/nlme/nlme.pdf}\\
    \bigskip
    \item \url{https://www.r-bloggers.com/2017/12/linear-mixed-effect-models-in-r/}\\
    \bigskip
    \item \textcolor{blue}{\url{https://rpsychologist.com/r-guide-longitudinal-lme-lmer}}\\
    \bigskip
    \item Finch, Bolin, Kelley: Multilevel Modeling Using R (2014).
\end{itemize}
\end{frame}
%---------------------------------------------------------------------
\end{document}